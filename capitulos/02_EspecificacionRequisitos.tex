\chapter{Especificación de requisitos}

En el siguiente capítulo se detallará la planificación temporal que ha seguido el proyecto, así como los recursos
utilizados y el presupuesto necesario para lograrlo.

\section{Planificación temporal}

\subsection{Estado del arte del trabajo}

\subsection{Análisis}

\subsection{Diseño e implementación}

\subsection{Pruebas}

\subsection{Redacción del documento de memoria}

\section{Recursos}

Para la realización del trabajo se han necesitado de 3 tipos de recursos, los cuales han estado presentes durante toda
la realización del trabajo: humanos, hardware y software.

\subsection{Recursos humanos}

Principalmente está compuesto por las personas involucradas en el desarrollo de la investigación. En este caso se trata
de dos figuras:

\begin{itemize}
    \item \textbf{José Luis París Reyes}, alumno del Máster en Formación Permanente en Ciberseguridad por la Escuela de 
      Postgrado de la Universidad de Granada en colaboración con la propia universidad, al que corresponde la autoría
      de este mismo documento.
    \item \textbf{D. Gabriel Maciá Fernández}, profesor perteneciente al departamento de Teoría de la Señal, Telemática
      y Comunicaciones, como tutor del proyecto, cuya funcionalidad principal ha sido la de dirigir al estudiante para la 
      investigación y estructura del mismo.
\end{itemize}

\subsection{Recursos hardware}

La realización de la investigación ha requerido de diferentes dispositivos hardware que se detallan a continuación:

\begin{itemize}
  \item \textbf{Ordenador de sobremesa personal}, en el que se han ejecutado todas las pruebas respectivas al trabajo,
    así como clonación de información y ataques a diferentes tipos de tarjetas. Este ha sido el dispositivo de mayor 
    prestaciones del que ha dispuesto el estudiante y el que ha otorgado mayor comodidad, por lo que la mayoría de 
    pruebas se han realizado sobre el mismo. Las características principales son:
    \begin{itemize}
        \item \textbf{Procesador: } Ryzen 5 5600X - 3700GHz - 6 Núcleos / 12 Hilos
        \item \textbf{Tarjeta Gráfica: } Nvidia RTX 3070 Ti
        \item \textbf{Memoria RAM: } 4x8GB DDR4 - 3200 MHz
    \end{itemize}
  \item \textbf{Ordenador portátil personal}, aunque no ha sido usado con mucha frecuencia porque se ha utilizado
    principalmente si el estudiante ha necesitado desplazarse a otro lugar de residencia o a una reunión con el tutor.
    Las características principales no serán detalladas porque su función principal ha sido la continuación en la 
    redacción del trabajo, por lo que sus prestaciones no han influido en la ejecución de las pruebas.
  \item \textbf{Conexión de fibra óptica a internet}, en todo momento se ha dispuesto de una conexión de alta velocidad
    a internet de 1Gbps. La función principal ha sido la de consulta bibliográfica, descarga y subida de ficheros desde
    el laboratorio de trabajo.
  \item \textbf{Lector de tarjetas RFID Proxmark3}, se trata del lector de tarjetas que se ha usado para realizar todas
    las pruebas del proyecto. \textbf{NO SÉ SI DEBERÍA INSERTAR LINK DE COMPRA}
  \item \textbf{Surtido de tarjetas RFID}, diferentes tarjetas que se han adquirido para la realización de 
    diferentes ataques o clonación de datos en función de la frecuencia o tipo de las mismas. 
    \textbf{NO SÉ SI DEBERÍA INSERTAR LINK DE COMPRA}
\end{itemize}

\subsection{Recursos software}

Aquel software del que se ha hecho uso y que será ampliado con mayor detalle en el capítulo 3.

\begin{itemize}
  \item \textbf{Sistema Operativo Linux}, ambos dispositivos (ordenador de sobremesa y ordenador portátil) hacen uso de
    una distribución de Linux basada en Arch Linux, en este caso se trata de la distribución Manjaro. Sobre este
    sistema operativo se han realizado todas las pruebas y se han instalado los diferentes programas.
  \item \textbf{Lenguajes de programación}, principalmente han sido lenguajes de scripting como \textit{Python} o
    \textit{Bash} para la automatización y ejecución de pruebas.
  \item \textbf{Firmware para Proxmark3}, se trata del firmware usado en la placa lectora de tarjetas y que se ampliará
    con más detalle en el capítulo 3.
\end{itemize}

\section{Presupuesto}

\chapter{Introducción}

\section {Motivación}

La evolución de la tecnología ha propiciado un gran avance para el ser humano y su calidad de vida, desde los primeros circuitos electrónicos y tarjetas perforadas en los años 60 y 70, hasta el supercomputador más sofisticado en la actualidad, capaz de hacer cálculos de gran complejidad en un tiempo inapreciable, identificación de patrones en ADN o predicción de fenómenos metereológicos, entre otras cosas. \newline

Toda esta tecnología se compone a grandes rasgos de un hardware y un software que le permite funcionar correctamente y satisfacer las necesidades de los usuarios. Dicha evolución ha propiciado la sofisticación de estos dos componentes, provocando que se vuelvan lo suficientemente complejos como para que no se puedan tener en cuenta todas las casuísticas por las que pueden fallar. \newline

El problema se vuelve aún mayor cuando esta tecnología, que permite almacenar, transmitir y procesar información y que está al alcance de cualquiera en forma de un dispositivo portable, es vulnerable, provocando que un usuario con intenciones malintencionadas obtenga información que no debería o adquiera el poder de realizar acciones suplantando la identidad de otra persona. \newline

Además, la popularización de la tecnología \textit{\acrshort{nfc}} que aparece en 2006 ha agravado esta situación, permitiendo establecer una comunicación inalámbrica de corto alcance y alta frecuencia para el intercambio de datos entre dispositivos \cite{NFCFechaCreacion}. Es decir, esta tecnología se encuentra en todas partes, desde los pagos con tarjetas de crédito o débito hasta la sincronización entre dispositivos móviles o la transferencia de información entre ellos. \newline

Es por ello que en este trabajo se revisarán las técnicas actuales que existen contra esa tecnología, se trabajará en posibles mejoras y se hará una realización práctica de las mismas en un laboratorio. La finalidad se basa en mostrar el peligro que supone la vulneración de su seguridad y lo importante que es la continua mejora en la protección de esta tecnología.

\section{Contexto}